\chapter{Často kladené otázky}


\subsection*{Kde nájdem svoj rozvrh?}

Na stránke \href{https://candle.fmph.uniba.sk}{\texttt{candle.fmph.uniba.sk}}
si môžeš vyhľadávať v rozvrhoch či vytvoriť vlastný. Zobrazený rozvrh je pre celú Tvoju skupinu (krúžok). Obsahuje ,,odporúčané" predmety, ktoré by si mal v danom semestri absolvovať. Pozor! v AiSe2 je síce ,,rozvrh", ale tento modul nie je na fakulte používaný.

\subsection*{Čo je to prerekvizita?}

V informačnom liste predmetu sa neraz vyskytuje označenie aj iného predmetu (tzv.\ prerekvizita). Táto notácia označuje fakt, že bez~absolvovania predmetu, ktorý je prerekvizitou, si daný predmet nemôžeš zapísať.
%sa daný predmet neodporúča zapísať (resp. po konzultácii s vyučujúcim).


\subsection*{Aké jazyky môžem na FMFI študovať?}

Angličtina, francúzština, nemčina a ruština. V~prípade záujmu o~iné
jazyky si môžeš zapísať predmety na inej fakulte univerzity (Pedagogická
alebo Filozofická fakulta). Zároveň musíš uzatvoriť tzv. \emph{Zmluvu o štúdiu} medzi našou fakultou a fakultou, ktorú chceš navštevovať. 


\subsection*{Môžem chodiť na prednášky, ktoré nemám zapísané?}

Áno, pokiaľ to nebude prekážať vyučujúcemu. Prednášky sú spravidla ve\-rej\-né
(či už u~nás, alebo na~inej fakulte).


\subsection*{Kde sa môžem dostať k počítaču?}

Počítače sa nachádzajú v učebniach H3, H6, M217, F1-248 a T3 (F2-128).
Vstup je možný iba s~preukazom študenta (karta ISIC). 

\subsection*{Je fakulta otvorená aj cez voľné dni (víkendy, sviatky)?}

Ak potrebuješ uniknúť ruchu internátov alebo domácej rutine, fakulta je Ti plne k dispozícii. Cez víkendy a voľné dni je otvorená od 7:00 do 22:00. Vstup je v týchto dňoch možný len cez fyzikálnu vrátnicu, na ktorej sa musíš \emph{zapísať}. Cez~víkend a voľné dni je Ti učebňa H3 k dispozícii od 9:00 do~19:00. Ostatné priestory (Vacuum lounge, Unispace) by mali byť otvorené v bežnom režime.  

\subsection*{Je na škole školský psychológ?}

Naša fakulta nezamestnáva psychológa, v~prípade potreby sa~však môžeš obrátiť na~Psychologickú poradňu pre~vysokoškolákov, ktorú zriadila Univerzita Komenského. Jej sídlo sa nachádza v~átriovom domku R. Viac sa dozvieš cez~email: 
\href{mailto:ppv@rec.uniba.sk}{\texttt{ppv@rec.uniba.sk}}.

\subsection*{Komu môžem nahlásiť drobné nedostatky technického charakteru (nefungujúce
svetlá, chýbajúce mydlo, poškodené sedenie a podobne)?}

Všetky technické problémy ohlás na prevádzku: \newline \href{mailto:prevadzka@fmph.uniba.sk}{\texttt{prevadzka@fmph.uniba.sk}}, kde sa promptne pokúsia Tvoj problém vyriešiť. 
