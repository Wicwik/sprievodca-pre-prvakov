\chapter{Práva a povinnosti študenta}

Je veľmi dobré a užitočné poznať svoje práva a povinnosti. V~prípade
štúdia na vysokej škole sú ich hlavnými zdrojmi:
\begin{enumerate}
\itemsep0em
\item zákon č. 131/2002 Z. z. o vysokých školách v znení neskorších predpisov,
\item Študijný poriadok fakulty a
\item smernice rektora univerzity či dekana fakulty.
\end{enumerate}
Odporúčame Ti prečítať si ich, my spomenieme len niektoré informácie
z~nich.


\section{Práva}

Medzi základné práva patrí právo študovať študijný program, na~ktorý
si bol(a) prijatý(á). Svoj študijný plán si môžeš v~medziach študijného
programu vytvoriť podľa svojho uváženia. \\

V~rámci svojho štúdia sa môžeš zároveň uchádzať aj o~štúdium na
inej vysokej škole, a to aj v~zahraničí. \\

Svoj názor nielenže môžeš slobodne prejaviť, ale dokonca budeme radi,
ak sa oň podelíš v \textbf{študentskej ankete} (pozri \ref{subsec:4.1.1}). Otázky, sťažnosti či pripomienky ohľadne Bc. a Mgr. štúdia môžeš napísať do~\textbf{čiernej skrinky} (pozri \ref{subsec:4.1.2}), ktorú nájdeš v AiSe. Vedenie fakulty si veľmi váži dobre mienenú (konštruktívnu) kritiku, snahu študentov o~akékoľvek zlepšenia
a pokiaľ je to možné, vyjde v~ústrety. \\

Ak si myslíš, že došlo k~porušeniu Tvojich práv, môžeš sa obrátiť v nasledujúcom poradí
 na viacero osôb. V~prvom rade sa snaž problém vyriešiť 
tam, kde vznikol (napr.\ priamo s~vyučujúcim). V~prípade neúspechu 
sa môžeš obrátiť na tútora alebo garanta študijného programu, potom na nás (ŠKAS) alebo na\emph{~prodekanku pre bakalárske a magisterské 
štúdium}. Poslednou inštanciou, na koho sa možno na fakulte obrátiť, je dekan.


\subsection{Študentská anketa} \label{subsec:4.1.1}

Už niekoľko rokov prebieha na fakulte na konci každého semestra (a počas skúškového obdobia) študentská anketa.
Študenti v~nej môžu \textbf{anonymne} vyjadriť svoj názor na~výučbu, celkový chod fakulty 
či technické vybavenie na~fakulte. \\

Študentskú anketu berie vedenie fakulty veľmi vážne. Snaží sa riešiť
problémy, na~ktoré študenti poukazujú. Preto Ti odporúčame zúčastniť
sa jej a pomôcť tak v~zlepšovaní kvality štúdia na našej fakulte. \\

Po skončení ankety vedenie uverejňuje tzv \emph{Stanovisko vedenia}, v ktorom sa snažia odpovedať na frekventované komentáre vo \emph{všeobecných otázkach}. Taktiež sa má možnosť vyjadriť učiteľ predmetu, či garant ku svojmu \emph{predmetu} alebo \emph{študijnému programu}. Výsledky ankety sa dajú využiť aj pri výbere predmetov. Po prihlásení
sa na stránke ankety uvidíš názory študentov na~kvalitu či obtiažnosť predmetov a ich odporúčanie, či si daný predmet zapísať, alebo nie. \\

Anketa sa nachádza na adrese \href{https://anketa.fmph.uniba.sk}{\texttt{anketa.fmph.uniba.sk}}.

\subsection{Čierna skrinka} \label{subsec:4.1.2}

Čierna skrinka je miesto pre otázky či pripomienky týkajúce sa
štúdia. Môžeš do nej anonymne alebo pod svojím menom napísať a na rovnakom mieste sa Ti bude
snažiť odpovedať prodekanka pre bakalárske a magisterské štúdium. Prosíme Ťa, nezneužívaj Čiernu skrinku ako ,,informačný kanál", na niektoré otázky Ti vieme odpovedať aj my \Smiley.\\

Nápady a potrehy, za ktoré sa nie je potrebné skrývať za rúšku anonymity, posielaj prodekanke pre bakalárske a magisterské štúdium mailom na: 

\begin{center}
\href{mailto:kristina.rostas@fmph.uniba.sk}{\texttt{kristina.rostas@fmph.uniba.sk}}
\end{center}

Čierna skrinka je prístupná cez AIS%
\footnote{\href{https://ais2.uniba.sk}{\texttt{ais2.uniba.sk}}%
} dvomi spôsobmi: 
\begin{enumerate}\setlength\itemsep{1pt}
\item Cez menu vľavo Administratívny systém > Diskusia k téme 
\item Cez oznam \textquotedbl{}Diskusie\textquotedbl{} v pravom stĺpci na
úvodnej stánke
\end{enumerate}

\subsection{FB stránka a skupina, YouTube kanál a Instagram}


%https://www.facebook.com/MatFyzJeIn/
K oficiálnym komunikačným médiam našej fakulty patrí Facebookova stránka \href{https://www.facebook.com/MatFyzJeIn/}{\texttt{MatFyzJeIn}} na ktorej nájdeš aktuálne udalosti na fakulte, eventy, informácie o úspechoch našich študentov a pedagógov a pod. Tiež odporúčame sledovať Facebookovu stránku \href{https://www.facebook.com/matfyzjobs/}{\texttt{matfyzjobs}}, na ktorej nájdeš uverejňované ponuky prevažne od absolventov našej fakulty.\\ 

V roku 2020 vznikla neformálnejšia Facebookova skupina s cieľom komornejšej komunikácie študentov, absolventov a priaznivcov MatFyzu. Tú nájdeš pod názvom ,,\emph{Fakulta matematiky, fyziky a informatiky - FMFI UK BA}". \\

Zaujímavé prednášky, video rozhovory a populárno-vedecké videá môžeš sledovať na YouTube
kanáli našej fakulty {\href{http://www.youtube.com/MatFyzJeIn}{\texttt{youtube.com\ /MatFyzJeIn}}%
}. \\

Ako ďalší zdroj informácií Ti môže poslúžiť aj fakultný Instagram @matfyjein. 
%K neformálnejším zdrojom informácii patria \href{http://blog.matfyz.sk}{\texttt{blog.matfyz.sk}}
%a \href{http://wiki.matfyz.sk}{\texttt{wiki.matfyz.sk}}. Na blogu si môžeš
%prečítať články ostatných študentov a na wiki nájsť rôzne informácie,
%ktoré študenti pokladali za užitočné. Informácie o prebiehajúcich
%podujatiach a novinkách, týkajúcich sa študentov nájdeš najmä na Facebookovej
%stránke\footnote{\href{https://www.facebook.com/MatFyzJeIn}{\texttt{facebook.com/MatFyzJeIn}}%
%}.



\section{Povinnosti}

Povinnosťou nás, študentov, je okrem snahy naučiť sa čo najviac aj
dodržiavanie vnútorných predpisov univerzity a fakulty. Medzi tieto
povinnosti patrí aj ochrana nášho spoločného majetku a dobrého mena
školy. V~prípade porušenia právnych predpisov alebo vnútorných predpisov
sa môžeš dostať aj pred \textbf{disciplinárnu komisiu} fakulty.\\

Všetky svoje práva a tiež povinnosti nájdeš \textbf{\href{https://zona.fmph.uniba.sk/fileadmin/fmfi/fakulta/legislativa/Studijny_poriadok_FMFI_UK_maj2020.pdf}{Študijnom poriadku fakulty}}, konkrétne čl. 2.
