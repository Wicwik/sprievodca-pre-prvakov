\chapter{Dianie na fakulte}

\section{Voľný čas na fakulte}

Počas teplejších dní môžeš ísť von do \textbf{Relaxu}. Na vrátnici
v~pavilóne matematiky Ti požičajú aj hojdacie siete, bedmintonové
rakety alebo discgolgové taniere, ktoré môžeš použiť na
\textbf{discgolfovom ihrisku} rozpriesterajúcom sa v okolí fakulty. Rovnako si môžeš požičať aj fixky na exteriérovú tabuľu.\\


Ako si si mohol počas svojho krátkeho času na MatFyze všimnúť, pri matematickej vrátnici (pod študijným oddelením, pozri mapu na str. \pageref{fig:mapa_fmfi}) vznikol v spolupráci s partnermi z praxe nový priestor na trávenie voľné času – UniSpace.\\

\textbf{Unispace} Ti poslúži kedykoľvek a v každom počasí, bude vybavená kaviarňou (s posedením), uzamykateľnými skrinkami, tromi ,,mítingovkami", relax a study zónou, pódiom na eventy, spoločenskými hrami a dokonca \textbf{funkčným klavírom} (slúchadlá k nemu si môžeš vypožičať na vrátnici). Ak chceš hrať bez slúchadiel, vždy musíš brať ohľad na ostatných ľudí v miestnosti, a študijné oddelenie nad~Tebou. Môžeš si v nej dobiť notebook alebo mobil (ak máš nabíjačku). V miestnosti taktiež nájdeš aj Wi-Fi. Ak sa nebudú v priestore konať žiadne eventy, priestor bude určený na štúdium a prácu pre študentov našej fakulty. \\

Pravidelne sa v nej budú konať eventy (semináre, workshopy) spoločností previazaných s praxou. Taktiež môžeš so spolužiakmi medzi vyučovaním používať \textbf{pingpongový stôl} za UniSpace-om, rakety a loptičky sa opäť požičiavajú na~vrátnici.

Ďalšou oddychovou miestnosťou bude \textbf{Vacuum lounge}. Nájdeš ju pri spojovacej chodbe medzi pavilónom posluchární a pavilónom fyziky (viď. mapu na str. \pageref{fig:mapa_fmfi}). V miestnosti nájdeš miesto na prácu s kapacitou cca 20 študentov, pohodlné gauče, tulivaky,  premietacie plátno s projektorom a klimatizáciu. Miestnosť bude primárne určená ako study zóna, v ktorej pravidelne bude spoločnosť Vacuumlabs (spolu s partnermi) organizovať workshopy, výstavy a pod. V ostatnom čase Ti bude plne k dispozícii na štúdium a trávenie času medzi prednáškami.\\

Za posluchárňou A sa nachádza priestor s tromi umiestnenými hojdacími sieťami a pohodlným gaučom. V pavilóne fyziky tiež nájdeš novovybudovanú oddychovo-pracovnú zónu so stolíkmi, zásuvkami a kreslami. Nachádza sa pri F1 246.\\

Na oddych ti bude slúžiť aj priestor pod schodiskom neďaleko miestnosti M-VII. Okrem stolov na štúdium pred prednáškou či cvičením v ňom nájdeš aj barové stoličky, či pohodlný gauč. \\

\textbf{Pavilón S} je pavilón športu na našej fakulte. Odohráva sa tu časť výuky telesnej výchovy. Nájdeš tu napr. ping-pongové stoly, bedmintonové /volejbalové ihrisko, lezeckú stenu, posilňovňu či dokonca aj saunu. Na poschodí sa nachádza miestnosť na jógu a nové sídlo ŠKASu. V prípade návštevy pavilónu mimo hodín telesnej výchovy platí nepísané pravidlo, že študenti majú zväčša vyhradené doobedné hodiny a zamestnanci poobedné hodiny. Treba sa však dohodnúť osobne alebo na tel. čísle 02/60295110 (klapka 110). %Lezecká stena mimo vyučovacích hodín prístupná nie je.

Voľné miestnosti na učenie či trávenie voľna so spolužiakmi môžeš
nájsť na~stránke \href{https://candle.fmph.uniba.sk}{\texttt{candle.fmph.uniba.sk}}. 


\section{Podujatia}

Počas roka sa na Matfyze okrem odborných podujatí odohrávajú aj oddychové
aktivity. V priebehu roka ŠKAS organizuje rôzne prednášky, napr. (prednáška Čo je to realita?, Karierny workshop, Time management atď.) \\

Obvykle sa koncom novembra koná \textbf{Beánia matfyzákov}. Ide o
spoločenskú udalosť, kde Ťa okrem prijímania novozapísaných študentov čaká kultúrny program, hudba a zábava.\\

V decembri organizujeme \textbf{Vianočný matfyzácky punč a kapustnicu}.
Táto akcia, ktorá sčasti oslavuje blížiaci sa koniec zimného semestra,
má aj príjemnú predvianočnú atmosféru.\\


Na Matfyze funguje taktiež iniciatíva \textbf{Život po matfyze}, ktorej členovia v priebehu roka organizujú rôzne spoločenské a edukačné aktivity. 

% @julia: toto je tiez zaujimava info :D mali by sme to organizovat?
% @andrej: o tom nič neviem a radšej by som to sem nedával, pretože potom by to mala byť pre nás už povinnosť :D
%Počas jari sme v posledných rokoch organizovali podujatie\textbf{
%Matfyz ožíva}, kde sa študenti môžu stretnúť s vedením a zamestnancami
%fakulty. Ponúka priestor pre rozhovory, hry, šport a naviac môžeme
%skrášliť naše fakultné prostredie. V~rámci nej vznikol aj levanduľový
%nápis pri soche Kopernika, vznikol priestor Relax a v ňom pribudli
%laná na~zavesenie hojdacích sietí, ohnisko, lavičky. V~takejto aktivite
%by sme radi pokračovali aj tento akademický rok.


\section{Športové vyžitie}

Po celodennom sedení určite oceníš aj aktívnejšiu zábavu. O~to sa
stará okrem iného aj Katedra telesnej výchovy a športu (KTVŠ), ktorá
Ti ponúka príležitosti športového vyžitia. Ponúkajú Ti telocvične,
ihriská a lodenicu. Novinky, oznamy, básničky a štýlových maskotov
nájdeš na~nástenke pri~vrátnici v pavilóne matematiky. \\

Pre nočné tvory fungujú počas celého roka tzv.\textbf{\ Večerné ligy},
ktoré sa odohrávajú v telocvični na~Mlynoch. V~priateľskej atmosfére
sa tu hrá basketbal, futsal, florbal a volejbal. \\

Každý semester sa pred skúškovým obdobím konajú \textbf{Športové dni
FMFI}, na~ktorých sa môžeš spolu s profesormi a naši absolventmi zapojiť 
do rôznych športových disciplín (je medzi nimi aj poker, čo je na Matfyze považované tiež za športovú disciplínu \Smiley{}).

\section{Študentské spolky}

\subsection{Študentská komora akademického senátu - ŠKAS}

Ako sme sa Ti na úvod tohto sprievodcu prestavili, sme skupinka 10-tich\footnote{Tento počet vychádza zo zákona o vysokých školách.} volených študentov, ktorí sa chcú podieľať na tvorbe lepšieho študijného prostredia na fakulte. Avšak to neznamená, že sa k nám nemôžeš pridať! Radi uvítame kohokoľvek so záujmom. 

\subsection{Trojsten}

Trojsten\footnote{\href{https://www.trojsten.sk}{\texttt{trojsten.sk}}} je občianske združenie, ktoré sa nachádza na našej fakulte. Samotný Trojsten sa delí na Korešpondenčný seminár z fyziky (FKS), Korešpondenčný seminár z matematiky (KMS) a Korešpondenčný seminár z programovania (KSP). Určite si sa počas strednej školy stretol s ich korešpondenčnými seminármi, zúčastnil si sa Letnej školy Trojstenu alebo si sa zúčastnil tzv. ,,Trojstenáckeho sústredka". \\

Ako v prípade ŠKASu, tiež ho tvoria dobrovoľníci. Vo svojom voľnom čase pomáhajú s tvorbou, opravou, písaním a konzultáciou zadaní a ,,vzorákov". Počas Tvojho pôsobenia v Trojstene získaš spomienky a vedomosti na celý život. \Smiley

\subsection{Študentský vývojový tím - ŠVT}

ŠVT\footnote{\href{https://svt.fmph.uniba.sk}{\texttt{svt.fmph.uniba.sk}}} má na našej fakulte dlhoročnú tradíciu. Tvoria ho tak zamestnanci, ako aj študenti. ŠVT stojí za úspešnými systémami, s ktorými sa stretneš (či dokonca už stretol) počas svojho štúdia na MatFyze. Za zmienku stojí Candle, Anketa, ePrihlaška, informačné obrazovky či VOTR. \\

Žiaľ, v dnešnej dobe sa im ťažko hľadajú nové posily, preto určite uvítajú, ak sa k nim pridáš. Neboj sa, každý raz začínal a máš príležitosť sa naučiť nové dovebnosti ,,priamo v teréne" pod dorozom najlepších informatikov v odbore. 

\subsection{Vocalatté – spievajúci nadšenci, nadšení speváci}

Vocalatté vzniklo v roku 2011 ako zoskupenie kamarátov z VŠ, ktorí zdieľajú spoločné nadšenie pre hudbu a spev. Ide o hudobných nadšencov, ktorí sa radi stretnú a participujú na rôznych hudobných projektoch. Stretávajú sa pravidelne, väčšinou jedenkrát týždenne vo večerných hodinách. Vocalatté momentálne disponuje 17 členmi, no noví členovia nikdy nie sú na škodu \Smiley. \\
Ak máš ďalšie otázky, neváhaj ich osloviť prostredníctvom FB alebo mailom\footnote{\href{mailto:vocalatte@gmail.com}{\texttt{vocalatte@gmail.com}}}.





