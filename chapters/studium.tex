\chapter{Štúdium}

%\begin{comment}
%Tvoje štúdium sa riadi podľa základného predpisu -- Študijného poriadku
%fakulty.
%\end{comment}
%\begin{comment}
%Základný predpis, podľa ktorého sa riadi Tvoje štúdium, je \textbf{Študijný
%poriadok fakulty}.
%\end{comment}



\section{Študijný poriadok fakulty}

Študijný poriadok fakulty (stiahneš na stránke študijného oddelenia%
\footnote{\href{https://zona.fmph.uniba.sk/studenti-a-studium/studijne-oddelenie//}{\texttt{zona.fmph.uniba.sk/studenti-a-studium/studijne-oddelenie/}}%
}) je základný predpis\textbf{, }podľa ktorého sa riadi Tvoje štúdium.
Vrelo Ti odporúčame si ho prečítať, získaš tak ucelenejší prehľad
o pravidlách a organizácii štúdia, než Ti môžeme poskytnúť v~rozsahu
tejto príručky.


\section{Študijné oddelenie}

Všetky potrebné informácie o Tvojom štúdiu, ako aj pomoc s~rôznou
nutnou byrokraciou, Ti isto počas úradných hodín rady poskytnú referentky študijného oddelenia. 

\begin{table}[h!]
\begin{centering}
\begin{tabular}{|c|c|c|}
  \hline
                 &  doobeda    	  & poobede        \\
  \hline
  \hline
    Pondelok     & -----------    & 13:00 -- 15:00 \\
  \hline
    Utorok       & -----------    & 13:00 -- 15:00 \\
  \hline
    Streda       & 9:00 -- 11:30  & 13:00 -- 15:00 \\
  \hline
    Štvrtok      & 9:00 -- 11:30  & -----------    \\
  \hline
    Piatok       &   -----------  & -----------    \\
  \hline
\end{tabular}
\par\end{centering}
\protect\caption{Úradné hodiny študijného oddelenia. Študijné oddelenie číslo dverí 5 má v utorok úradné hodiny doobeda, nie poobede.} %Musis overit
\label{tab:SO-UH}
\end{table}

Študijné oddelenie sa nachádza v~pavilóne matematiky (pozri mapu
na str.\ \pageref{fig:mapa_fmfi}).

Každý študijný program má pridelenú svoju re\-fe\-ren\-tku (pozri
menovky pri dverách). Potrebné tlačivá a potvrdenia nájdeš na~fakultnom
webe\footnote{\href{https://zona.fmph.uniba.sk/sluzby-a-administrativa/tlaciva/}{\texttt{zona.fmph.uniba.sk/sluzby-a-administrativa/tlaciva/}}}. Dôležité oznamy študijného oddelenia nájdeš na nástenke pred študijným oddelením (oznamy o rektorských a dekanských voľnách, oznamy o odborových a prospechových štipendiách) alebo na Facebookovej stránke ,,Študijné oddelenie FMFI UK"\footnote{\href{www.facebook.com/fmphstudijne/}{\texttt{facebook.com/fmphstudijne/}}} \\

V prípade zlej epidemiologickej situácie, Študijné oddelenie poskytovalo služby výhradne telefonicky a mailovou komunikáciou. V tomto prípade je potrebné s referentkami komunikovať prostredníctvom univerzitne pridelenej mailovej adresy (viď strana \pageref{sec:6.1}). 


\section{Harmonogram štúdia}

Slávnostné otvorenie akademického roka\footnote{Účasť na otvorení akademického roku je dobrovoľná.} {\akademickyRok} sa uskutoční {\otvorenieARd} o {\otvorenieARh} v Aule UK (Šafárikovo námestie č. 6, Bratislava).

\begin{longtable}{|>{\centering}p{0.33\textwidth}|>{\raggedright}p{0.59\textwidth}|}
  \hline 
    \multicolumn{2}{|l|}{\textbf{Zimný semester}}\tabularnewline
      \hline 
      2. 9. – 3. 9. 2021       & Zápis študentov v 1. roku Bc. štúdia\tabularnewline
      \hline 
      27. 8. – 17. 9. 2021     & Zápis študentov vo vyššom roku Bc.\ a Mgr.\ štúdia\tabularnewline
      \hline 
      {20. 9. – 17. 12. 2021}  & {Výučba FMFI (13 týždňov)}\footnote{Študenti odborov, ktorí majú výučbu na viacerých fakultách, môžu mať odlišné termíny výučby. \label{fn2}}\tabularnewline
      \hline 
      koniec septembra         & Slávnostná imatrikulácia novoprijatých študentov\tabularnewline
      \hline 
      koniec novembra          & Beánia Matfyzákov\tabularnewline
      \hline 
      december                 & Vianočná kapustnica a punč\tabularnewline
      \hline 
      3. 1. – 11. 2. 2022      & Skúškové obdobie (7 týždňov) \tabularnewline
      \hline 
    \hline 
    \multicolumn{2}{|l|}{\textbf{Letný semester}}\tabularnewline
      \hline 
      14. 2. – 13. 5. 2022     & Výučba FMFI (13 týždňov)$^\ref{fn2}$ \tabularnewline
      \hline 
      14. 4. – 19. 4. 2022     & Rektorské / dekanské voľno (veľkonočné sviatky) \tabularnewline
      \hline 
     % do 19. 2. 2020           & Odovzdanie indexov prvákov na kontrolu ŠO\tabularnewline
     %\hline
      14. 5. – 30. 6. 2021      & Skúškové obdobie (6 týždňov a 4 dni)\tabularnewline
      \hline 
      júl & Podávanie žiadostí o ubytovanie\tabularnewline
    \hline
\end{longtable}


\section{Dĺžka štúdia}

Štandardná dĺžka štúdia na našej fakulte sú 3 roky pre bakalárske študijné programy, resp. 4 roky pri tzv. konverzných študijných programoch (FYZ, OZE, TEF, DAV a BIN) a 2 roky pre~magisterské štúdium, resp. 3 roky pre konverzné magisterské študijné programy (napr. učiteľské ŠP, MPG, INF a AIN). Prekročenie štandardnej dĺžky štúdia je spoplatnené školným.

Štúdium je možné (podľa Študijného poriadku fakulty) prerušiť naj\-viac
na 1 rok, pri závažných dôvodoch naj\-viac na~2 roky a pri~rodičovskej
dovolenke najviac na 3 roky. Avšak na dané obdobie strácaš štatút študenta.


\section{Kreditový systém}

Kre\-di\-tový systém štúdia využíva zhromažďovanie a prenos \emph{kre\-di\-tov}
(napr. medzi fakultami či univerzitami). Kredity sú číselné hodnoty
priradené k~predmetom, vyjadrujúce množstvo práce potrebnej na ich
absolvovanie. \\

Priemerná záťaž študenta za~celý akademický rok je 60 kreditov,
za semester 30 kreditov. Maximálna záťaž v jednom roku je 90 kreditov
(výnimky s povolením dekana). Študent získava kredity po úspešnom absolvovaní predmetu (t.j. po obdržaní hodnotenia A až E). Počet získaných kreditov za predmet nie je závislý od získaného hodnotenia. \\

Na úspešné ukončenie trojročného bakalárskeho štúdia potrebuješ získať
180 kreditov (240 kreditov v prípade konverzných ŠP). V prípade dvojročného magisterského štúdia je to 120 (resp. 180) kreditov.

\section{Kontrolné etapy štúdia}

Na konci prvého (zimného) semestra je potrebné, aby si získal minimálne \textbf{20 kreditov} za úspešne absolvované predmety. Minimálny počet získaných kreditov, reprezentujúci tzv.\ mi\-ni\-málne tempo štúdia, je \textbf{40 kreditov} za~akademický rok s výnimkou prípadu, že študentovi ostáva vykonať len štátnu skúšku.

\section{Zápis predmetov}

Každý študijný program má svoj \textbf{odporúčaný študijný plán},
v~ktorom je pre~každý semester uvedený zoznam povinných (\textbf{A}), povinne
voliteľných (\textbf{B}) a výberových (\textbf{C}) predmetov. Štandardne si zapisuješ predmety
podľa tohto plánu. Všetky plány sú prístupné na stránke fakulty%
\footnote{\href{https://zona.fmph.uniba.sk/studenti-a-studium/studijne-programy/}{\texttt{zona.fmph.uniba.sk/studenti-a-studium/studijne-programy/}}%
}, kde nájdeš štruktúru (osnovu) predmetov, ich hodnotenie, zoznam prerekvizít, dokonca sylaby štátnych skúšok. 

Okrem prezretia  informačných listov Ti zároveň odporúčame zúčastniť sa prvých prednášok, na ktorých je vyučujúci povinný Ťa oboznámiť s východiskovou literatúrou, sylabami predmetu, prezenciou a hodnotením.

V~prípade záujmu si môžeš zapísať aj predmety určené pre~iný rok
štúdia či z~iných študijných programov. Pred zápisom takýchto predmetov
Ti ale odporúčame pozrieť si ich prerekvizity a konzultovať náročnosť
s~tútorom či samotným učiteľom. \\

Môžeš si zapísať aj \textbf{mimofakultné predmety} v~rámci UK či
inej univerzity. V~prvom rade sa ale treba dohodnúť s~príslušným
učiteľom. Kvôli prenosu kreditov treba následne komunikovať s prodekanmi
pre bakalárske a magisterské štúdium oboch fakúlt -- mali by podpísať tlačivo
„Zmluva o~štúdiu“. \\

Samotný zápis predmetov (na celý akademický rok) prebieha elektronicky
cez akademický systém AiS2, resp. cez VOTR. Na~študijnom oddelení Ti následne vystavia protokol o študijnom pláne. Prvé dva týždne každého semestra je možné zmeniť si už zapísané predmety z dôvodu napr.\ kolízie v rozvrhu.

\section{Absolvovanie predmetov}

Predmet sa považuje za úspešne \emph{absolvovaný}, ak splníš aspoň minimálne kritériá
na jeho splnenie. To znamená, že musíš z~neho dosiahnuť hodnotenie známkou A až E. Zároveň študent získa kredity len za úspešne absolvované predmety.

Hodnotenie predmetov sa spravidla skladá z~hodnotenia za~semester
a z~hodnotenia záverečnou skúškou. Niektoré predmety sú hodnotené
len jednou z~týchto zložiek. \\

Základné pravidlo je, že za svoje štúdium musíš absolvovať všetky
povinné a výber z povinne voliteľných predmetov podľa pravidiel určených
študijným programom (väčšinou treba nazbierať určitý počet kreditov
zo~skupiny predmetov). \\

Ak sa Ti nepodarí úspešne absolvovať povinný predmet, musíš si ho znova zapísať.
Povinne voliteľný predmet si môžeš aj nahradiť. Výberové predmety
nie je potrebné opakovať; ak sa tak ale rozhodneš, musíš ich úspešne
absolvovať. \\

V rámci skúškového obdobia máš okrem štandardného termínu skúšky nárok
aj na~dva \textbf{opravné termíny}, ak spĺňaš podmienky priebežného hodnotenia predmetu (podmienky si určuje každý vyučujúci sám). Pri~\textbf{opakovanom} zápise predmetu máš nárok taktiež na \textbf{dva} opravné termíny.

\section{Preukaz študenta (karta ISIC)}

Preukaz študenta je kombinovaný s~kartou ISIC. Jednu kartu tak používaš
ako preukaz študenta, preukaz do~Univerzitnej / fakultnej knižnice, električenku, zľavu a peňaženku v~internátnych jedálňach a pod. Ku ISIC-u sa viaže množstvo zliav, preto ak plánuješ niekam cestovať, navštíviť kino, nakupovať, atď., nezabudni sa dopredu informovať
o~výhodách na stránke \href{http://www.isic.sk}{\texttt{isic.sk}}.

\section{Knižnice a copycentrá}

\textbf{Fakultná knižnica so študovňou} sa nachádza na -1. poschodí pavilónu
informatiky (pozri mapu na str.\ \pageref{fig:mapa_fmfi}). Prezenčné
knihy môžeš študovať v študovni alebo si ich môžeš vypožičať na večer, či víkend.
Ostatné knihy si môžeš požičať až na~120~dní. 

Ak potrebuješ knihu na obdobie dlhšie ako 120 dní, môžeš si výpožičnú dobu predĺžiť ešte 2-krát (dovedna máš k dispozícii až jeden technický rok). Môžeš tak urobiť prostredníctvom online katalógu Knižničného systému Akademickej knižnice UK\footnote{\href{http://alis.uniba.sk:8088}{\texttt{alis.uniba.sk:8088}}%
}, osobne, písomne (mailom\footnote{\href{mailto:kniznica@fmph.uniba.sk}{\texttt{kniznica@fmph.uniba.sk}}}) alebo telefonicky.

Knižnicu nájdeš aj na FB\footnote{\href{https://www.facebook.com/kniznicafmfi}{\texttt{facebook.com/kniznicafmfi}}}, kde pravidelne informujú o rôznych podujatiach zo sveta vedy, ktoré by študentov mohli zaujímať. Tiež im môžeš zanechať správu na FB a radi Ti pomôžu.\\

Knihy ku konkrétnemu predmetu si môžeš vyhľadať v online katalógu podľa krátkeho kódu predmetu (napr. 1-INF-210) či podľa autora alebo názvu knižky. Ak je kniha vypožičaná, môžeš si ju pomocou online katalógu vyžiadať (tzv. žiadankou), čím sa skráti výpožičná lehota tomu, kto ju má požičanú a kniha Ti bude k dispozícii spravidla do 7 dní. Keď ju študent vráti, Tebe príde automaticky mail, že je pripravená v knižnici. Treba si ju vyzdvihnúť do 7 dní. Video sprievodcu knižnicou a knižničným systémom môžeš nájsť na linke \href{http://video.matfyzjein.sk/kniznica}{\texttt{video.matfyzjein.sk/kniznica}}}. \\ %
}%príp. pomocou QR kódu:

%\begin{center}
%	\vspace{-0.1cm}
%	\includegraphics[width=0.4\textwidth]{obrazky/qr_code_LIB.png}
%	\par\end{center}

Na blížiaci sa koniec výpožičnej lehoty Ťa upozorní e-mail na Tvoju univerzitnú adresu (viac sa o nej dočítaš v sekcii \ref{sec:6.1}). Je dôležité vracať knihy načas. Okrem toho, že tak pomôžeš spolužiakom sa ku knihe dostať, vyhneš sa aj poplatkom za omeškanie. Pokuta nie je zanedbateľná, a to 50 centov za knihu za deň. \\

Ako študent UK máš prístup k elektronickým študijným materiálom, k externým informačným zdrojom - databázam, článkom atď. Všetky informácie nájdeš na stránke\footnote{\href{https://zona.fmph.uniba.sk/sluzby-a-administrativa/kniznicne-sluzby/}{\texttt{zona.fmph.uniba.sk/sluzby-a-administrativa/ kniznicne-sluzby/}}} alebo priamo v knižnici, kde vždy nájdeš najnovšie informácie. Zároveň si môžeš požiadať aj o vzdialený prístup a študovať materiály z pohodlia domova. \Smiley

\begin{table}[h!]
\begin{centering}
\begin{center}
\begin{tabular}{|c|c|}
\hline 
Po - Štv & 8:30 -- 15:00 \tabularnewline
\hline 
Pia     & 8:30 -- 13:00 \tabularnewline
\hline
\end{tabular}
\par\end{center}

\par\end{centering}
\protect\caption{Otváracie hodiny knižnice počas akademického roka (od 1.9 - 30.6). Každý posledný piatok v mesiaci je knižnica zatvorená z~dôvodu sanitárneho dňa.}
\label{tab:uradne_hodiny}
\end{table}

Ak si potrebuješ nejakú knižku vypožičať/pozrieť i počas leta (napr. kvôli bakalárke), knižnica je otvorená od pondelka do piatku od 8:30 do 12:30.\\

\noindent
V knižnici sa nachádza aj samoobslužná kopírka/tlačiareň na~mince s\,cenou 5 centov za~stranu A4. Toto zariadenie je napojené na~PC, v~ktorom nájdeš vzorové tlačivá niektorých potvrdení a žiadostí. Možná je aj tlač z~USB kľúča alebo po prihlásení priamo zo súborov uložených na Tvojom fakultnom konte.

Niektoré skriptá sa dajú zakúpiť v~copycentre \textbf{PACI} na~Prírodovedeckej
fakulte. Môžeš si tu nechať aj zviazať svoju prácu. Ďalšie copycentrá
nájdeš na internátoch a v Cubicone.

\begin{comment}
Rôzne skriptá môžeš nájsť a následne si aj vziať v \textbf{skriptárni}, t.j. na označenej poličke pred knižnicou, kde môže ktokoľvek pre druhého študenta zanechať svoje už nepotrebné skriptá.
\end{comment}

Knižnica v \textbf{Centre vedecko-technických informácií}%
\footnote{\href{http://www.cvtisr.sk/}{\texttt{cvtisr.sk}}%
} sa nachádza blízko zastávky \emph{Patrónka}. Za malý ročný poplatok máš prístup k veľkému množstvu publikácií, skrípt a pomocou vzdialeného prístupu do e-zdrojov sa môžeš dostať k rôznym zahraničným článkom. Ak knižku nenájdeš vo fakultnej knižnici, pravdepodobne sa bude dať požičať tu.

Na Michalskej ulici sa nachádza \textbf{Univerzitná knižnica}%
\footnote{\href{http://www.ulib.sk/sk/}{\texttt{ulib.sk/sk/}}%
}. Je to najstaršia a najväčšia vedecká knižnica v Slovenskej republike.
Knihy sa v nej objednávajú pomocou internetového katalógu. 

\section{Mobilita}

Počas štúdia môžeš absolvovať zahraničný študijný pobyt alebo stáž.
Veľmi rozšírený program v rámci EÚ, umožňujúci obe formy, je program
\textbf{Erasmus}, resp. \textbf{Erasmus+}. Ďalej je tu \textbf{Národný štipendij\-ný program},
granty na~základe bilaterálnych dohôd či granty zo~súkromného sektora.
Viac informácií nájdeš na~stránkach fakulty alebo agentúry SAIA\footnote{\href{https://www.saia.sk}{\texttt{www.saia.sk}}%
}.

Vďaka mobilite môžeš študovať či pracovať v zahraničí, ale najmä získať
nové skúsenosti a priateľstvá, zlepšiť sa v cudzom jazyku (zväčša
v angličtine), či spoznať novú kultúru.

\section{Štipendiá}

Štipendiá sa riadia zákonom o vysokých školách, ktorý dopĺňa \textbf{Štipendijný
poriadok fakulty\footnote{\href{https://zona.fmph.uniba.sk/fileadmin/fmfi/fakulta/legislativa/Stipendijny_poriadok_FMFI_UK_uplne_znenie_jun2020.pdf}{\texttt{zona.fmph.uniba.sk/fileadmin/fmfi/fakulta/legislativa/\-Stipendijny\_poriadok\_FMFI\_UK\_uplne\_znenie\_jun2020.pdf}}}}. Tam sú uvedené aj konkrétne kri\-té\-riá ich prideľovania. Štipendiá sa rozdeľujú na motivačné, odborové a sociálne. 

\subsection{Motivačné štipendium}

Motivačné štipendium sa udeľuje za~vynikajúci prospech (\emph{prospechové štipendium}) alebo za mimoriadne výsledky (\emph{mimoriadne štipendium}).

\subsubsection{Prospechové štipendium}
Prospechové štipendium je priznávané 10\% študentov (tvoria sa 2 poradovníky: pre prvákov a pre druhákov a vyššie roky štúdia). Na to, aby si ako prvák dostal v letnom semestri prospechové štipendium, musíš splniť nasledujúce podmienky: 

	\begin{enumerate}
		\itemsep0em 
		\item vo všetkých zapísaných predmetoch za zimný / letný semester si získal hodnotenie A až E, 	
		\item dosiahol si vážený študijný priemer menší alebo rovný 1,30,
		\item a získal si najmenej 27 kreditov za semester.
	\end{enumerate}

\subsubsection{Mimoriadne štipendium}

Mimoriadne štipendium je udeľované za dosiahnutie vynikajúceho výsledku v oblasti štúdia, výskumu, vývoja, umeleckej alebo športovej činnosti.

\subsection{Odborové štipendium}

Odborové štipendium sa udeľuje časti študentov\footnote{50\% študentov 1. ročníka Bc. a vo vyšších ročníkoch je odborové štupendium priznané 30\% študentom} na študijných odboroch podporovaných Ministerstvom školstva, vedy, výskumu a športu SR (všetky odbory na FMFI UK sú podporované). 

\subsection{Sociálne štipendium}

Sociálne štipendium môže byť priznané iba tým študentom, ktorých rodinné príjmy sa približujú sumám životného minima. Posudzovať sa môžu príjmy Tvojich rodičov, súrodencov, manžela/manželky, príp. vlastných detí - teda najbližší rodinní príslušníci. Posudzuje sa vždy aktuálna situácia v čase podania žiadosti. Ak je štipendium priznané, vypláca sa v mesačných čiastkach od mesiaca, v ktorom bola žiadosť podaná. \\

Sociálne štipendiá má vo svojej agende p. Mináriková\footnote{\href{mailto:Maria.Minarikova@fmph.uniba.sk}{\texttt{Maria.Minarikova@fmph.uniba.sk}}} (dvere č.4) a motivačné, odborové, prospechové štipendiá rieši p. Majerčíková\footnote{\href{mailto:Emilia.Majercikova@fmph.uniba.sk}{\texttt{Emilia.Majercikova@fmph.uniba.sk}}} (dvere č.2) .\\

Viac informácií ohľadom štipendií nájdeš na stránke\footnote{\href{https://zona.fmph.uniba.sk/studenti-a-studium/stipendia/}{\texttt{zona.fmph.uniba.sk/studenti-a-studium/stipendia/}}} alebo u svojej študijnej referentky.

\newpage
\section{Študentská pôžička}

V prípade potreby môžeš požiadať o študentskú pôžičku z Fondu na podporu
vzdelávania vo výške až 5 000 € (pre doktorandov 8 000 €) na akademický
rok. Splátky s úrokom vo výške 3\% sa začnú až po skončení štúdia.
Lehota splatnosti je najviac 10 rokov. Viac sa dozvieš na \href{http://www.fnpv.sk}{\texttt{fnpv.sk}}.
