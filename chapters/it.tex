\chapter{Informačné technológie}

\section{Univerzitná e-mailová adresa} \label{sec:6.1}

Každý študent Univerzity Komenského v Bratislave má pridelenú e-mailovú
adresu v~tvare:

\begin{center}
\texttt{<prihlasovacie\_meno>@uniba.sk}.
\end{center}

\noindent
Prihlásiť do webmailu sa môžeš na adrese: \href{https://www.outlook.com/uniba.sk}{\texttt{outlook.com/uniba.sk}}. Alebo menej efektívnym spôsobom cez \href{https://www.outlook.com}{\texttt{outlook.com}}, kde zadáš svoju e-mailovú adresu v hore uvedenom tvare.

Okrem 20~GB e-mailovej schránky máš možnosť inštalovať si produkty \emph{Microsoft Office} na svoje počítače\footnote{licencia umožňuje inštaláciu až na \emph{piatich} počítačoch} (pozri sekciu \ref{sec:7.6}), prístup ku veľkému cloudovému úložisku a k ďalším službám cloudovej platformy \emph{Office 365}. \\

Prihlasovacie meno a heslo je rovnaké ako do AiSu. Máš povinnosť čítať si poštu prichádzajúcu na túto e-mailovú adresu, budú Ti tam chodiť oficiálne oznamy od vyučujúcich, fakulty či ŠKASu. Ak si poštu neprečítaš, neospravedlňuje Ťa to a môže sa stať, že kvôli tomu napríklad zaplatíš vyššiu pokutu v knižnici. Aby si nepremeškal žiadnu dôležitú poštu, odporúčame Ti presmerovať si fakultný mail na Tvoj súkromný. Podrobný video návod nájdeš pomocou QR kódu: 

%

\begin{center}
	\includegraphics[width=0.4\textwidth]{obrazky/qr_code_IT.png}
	\par\end{center} 

Ak nevieš svoje heslo, pomôže Ti správca hesiel na študijnom oddelení.

\section{AiS2} 

AiS2 - Akademický informačný systém bude Tvojím nevyhnutným spoločníkom počas štúdia. Bude Ti potrebný pri zápise predmetov (okrem papierového indexu), na prihlasovanie sa na skúšky / záverečné práce či ako vstupná brána do Čiernej skrinky, pozri v sekcii \ref{subsec:4.1.2}. \\

AiS2 je komplikovanejší ako sa na prvý pohľad možno zdá, preto Ti odporúčame, aby si sa ešte na začiatku Tvojho štúdia s ním oboznámil, vyvaruješ sa tak zbytočným komplikáciam. Tento pocit umocňuje jeho rozhranie či ,,atypické" tlačidlá (bežci či kvetinky). Ak si nebudeš dať vedieť s niečím rady, navštív stránku\footnote{\href{https://uniba.sk/o-univerzite/fakulty-a-dalsie-sucasti/cit/citps/ais/prirucky-a-navody/}{\texttt{uniba.sk/o-univerzite/fakulty-a-dalsie-sucasti/cit/citps/ais/prirucky-a-navody/}}}, na ktorej nájdeš praktické (obrázkové) návody. \\

Preto sa rozhodol ŠVT Ti uľahčiť trápenie a vyvinul vlastnú ,,light" verziu AiSu - \textbf{Votr}. 

\subsection{Votr}

Votr pochádza z dielne ŠVT. Je to prehľadnejšia verzia AiS2. Prihlasuješ sa doň rovnakými prihlasovacími údajmi ako do AiS2. \\

Votr nájdeš na adrese: 


\begin{center}
	\href{https://votr.uniba.sk}{\texttt{https://votr.uniba.sk}}.
\end{center}

Cieľom Votr nie je absolútna eliminácia AiS2, ale uľahčenie vykonávania základných úkonov, napr. kontrola kreditov, prihlasovanie sa na skúšky, zápis predmetov, vyhľadávanie študentov/učiteľov a predmetov. Ak potrebuješ vykonať niečo komplexnejšie, musíš použiť AiS2.  

\section{Fakultná Wi-Fi sieť}

Fakultná Wi-Fi sieť pokrýva veľkú časť fakulty, najmä frekventované
chodby a učebne. Wi-Fi sieť je súčasťou siete \emph{Eduroam}, ktorá je dostupná
aj na iných fakultách UK, univerzitách u nás a~v~zahraničí. \\

Počas doby používania wifi siete (ale aj počítačov v škole) si povinný dodržiavať \textbf{\href{https://zona.fmph.uniba.sk/fileadmin/fmfi/fakulta/legislativa/Vseobecne\_pravidla\_spravania\_sa\_pouzivatela\_informacnych\_technologii\_na\_FMFI\_UK.pdf}{Všeobecné pravidlá správania sa používateľa počítačovej siete}}. \\


 Prístupové údaje spolu s~návodmi nájdeš na~adrese \href{https://www.uniba.sk/wifi}{\texttt{uniba.sk/wifi}}.

Problémy s prístupom do počítačovej siete a~s~ďalšími IT službami môžeš riešiť prostredníctvom fakultného HelpDesk-u:

\medskip\noindent
e-mailom: \href{mailto:helpdesk@fmph.uniba.sk}{\texttt{helpdesk@fmph.uniba.sk}},\\
osobne: M-169 v pondelok až štvrtok 11:00-14:00\\
alebo telefonicky: 02/602 95 842.

\section{Počítačové učebne a študentský klaster}

Všetky počítače v počítačových učebniach majú prakticky identický
softvér, majú nainštalované operačné systémy \emph{Windows 7} a \emph{Debian Linux}, bežný kancelársky softvér a softvér potrebný na výučbu jednotlivých predmetov. Mimo vyučovacích hodín v~rozvrhu je ku počítačom voľný
prístup, na otvorenie miestnosti použiješ svoj ISIC. Prihlasuješ sa
pomocou rovnakého používateľského mena a hesla ako do AiS-u. Akékoľvek problémy, prosím, okamžite hlás správcovi, aby mohli byť vyriešené.

\medskip\noindent
Učebne: I-H3, I-H6, M-208, M-217, F1-248, T3 (F2-128).\\
Správca: Miroslav Wagner, I-22,\\
\href{mailto:miroslav.wagner@fmph.uniba.sk}{\texttt{miroslav.wagner@fmph.uniba.sk}}, 02/602 95 111

\medskip\noindent
Okrem počítačov v učebniach sa môžeš prihlásiť do~školy aj cez~vzdialený prístup na svoje konto na študentskom klastri da~Vinci. Na~klastri sa ukladajú tvoje súbory (sú prístupné aj na počítačoch v
učebniach, takže nemusíš všetko prenášať na USB kľúčoch), môžeš si tam zriadiť vlastnú web stránku, môžeš tam spúšťať softvér viazaný na školské licencie (napríklad Matlab) a riešiť domáce úlohy pod operačným systémom Linux.

\medskip\noindent
Prístup na klaster cez \emph{ssh} (môžeš použiť putty, winscp),\\
adresa: \href{http://davinci.fmph.uniba.sk/}{\texttt{davinci.fmph.uniba.sk/}},\\
tvoja stránka: \texttt{www.st.fmph.uniba.sk/~<prihlasovacie\_meno>}.\\
Správca: Matej Zagiba, F1-115,\\
\href{mailto:matej.zagiba@fmph.uniba.sk}{\texttt{matej.zagiba@fmph.uniba.sk}}, 02/602 95 127

\section{Web stránka fakulty}

Ku web stránke fakulty pristupuješ pomocou adresy:

\begin{center}
	\vspace{-0.12cm}
\href{https://zona.fmph.uniba.sk}{\texttt{zona.fmph.uniba.sk}}.
	\vspace{-0.12cm}
\end{center}

\noindent
Toto je stránka určená pre zamestnancov a študentov. V~ľavom dolnom paneli nájdeš odkazy na~často používané služby (rozvrhy Candle, webmail, AiS2, alternatívne rozhranie k AiSu - VOTR a pod.). \\

Vedľa potom nájdeš \emph{Správy z informačnej tabule}. Informačné tabule sú obrazovky rozmiestenené neďaleko hlavných vstupov (vrátnic). Informujú o zaujímavých akciách na fakulte (semináre, prednášky, spoločenské eventy a pod.), ak sa Ti ,,neušla" zaujímavá informácia, tu ju nájdeš \Smiley. V pravom hornom rohu nájdeš aj odkazy na fakultnú Facebook stránku a YouTube kanál. 

\section{Softvér} \label{sec:7.6}

Ako študent našej fakulty máš prístup do systému Microsoft Dreamspark
Premium na adrese
\href{https://moja.uniba.sk/msdn}{\texttt{moja.uniba.sk/msdn}}. Na tejto stránke
si zadarmo môžeš stiahnuť nové operačné systémy a vývojárske nástroje
firmy Microsoft. Môžeš ich využívať na nekomerčné účely a môžeš si ich
nechať nainštalované aj po skončení štúdia.

Na adrese \href{https://portal.office.com/Home}{\texttt{https://portal.office.com/Home}} si v môžeš zadarmo stiahnúť kancelársky balík \emph{Office 365}. Ten zahŕňa všetky známe kancelárske programy (Excel, Word, powerpoint) ale aj Skype for business či Access. Okrem toho je Ti k dispozícii aj 5 TB v cloudovom úložisku \emph{OneDrive}.

